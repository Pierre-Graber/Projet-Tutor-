\documentclass{article}
\usepackage[utf8]{inputenc}
\usepackage{amsmath}
\usepackage{amsfonts}

\title{Projet Tutoré : Licence Maths-Info
\section*{Génération de grands nombres premiers aléatoires}
\date{ Année 2019-2020}
}


\author{Pierre GRABER, Elias DEBEYSSAC, Toky RANDRIAMALALA }


\begin{document}

\maketitle

\section*{Introduction}
Les nombres premiers sont des nombres mystérieux que les mathématiciens étudient depuis des siècles tant pour leurs propriétés algébriques que pour le caractère semblant aléatoire de leur apparition dans l'ensemble ordonné des entiers naturels, $\mathbb{N}$. En effet les nombres premiers sont largement utilisés en arithmétique et en cryptologie, car certaines de leurs propriétés permettent de garantir la sécurité de systèmes cryptographiques exploitant des problèmes mathématiquement très difficiles à résoudre, qui nécessiteraient des années de calcul par les ordinateurs les plus puissants utilisés de nos jours. \newline 
La sécurité de ces systèmes de cryptage repose par exemple sur la difficulté de factoriser en nombres premiers de très grands nombres. De nos jours les ordinateurs, téléphones, cartes à puces utilisent une quantité industrielle de nombres premiers afin d'assurer des méthodes de cryptage fiables. La génération de grand nombres premiers est donc indispensable pour la sécurité des systèmes informatiques. Ce projet tutoré a pour but d'implémenter et d'expérimenter des algorithmes de génération de grands nombres premiers efficaces, et utilisables à des fins cryptographiques. Dans un premier temps les algorithmes seront implémentés dans un langage proche de celui de Python, grâce au logiciel de calculs mathématiques SageMath, tous les algorithmes seront tirés du livre "Handbook of Applied cryptography".

\section*{Plan}
\section*{Les tests de Primalité}
On ne connait pas de formule donnant la totalité des nombres premiers ou permettant de calculer le "n-ième" terme de la suite des nombres premiers. C'est pourquoi les générateurs de nombres premiers aléatoires utilisent des processus de choix de nombres aléatoires et testent la primalité de ces nombres. La répartition des nombres premiers nous assure qu'en effectuant de manière récursive un tel algorithme nous finirons par tomber sur un premier.



\section*{L'aléatoire de Python (donc de Sage)}
 
  



\end{document}